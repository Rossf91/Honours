\documentclass[a4paper,final,oneside,12pt]{report}
\usepackage[latin1]{inputenc}
\usepackage[dvips]{color}
\usepackage{fancyhdr}
\usepackage[bookmarksopen=true,bookmarks=true,pdfpagemode=UseOutlines,a4paper]{hyperref}
\usepackage{graphicx}
\usepackage{cite}
\usepackage{ifthen}
\usepackage{everyshi}
\usepackage{setspace}
%% Control the fonts and formatting used in the table of contents.
\usepackage[titles]{tocloft}
\usepackage{tocbibind}
\usepackage[flushmargin, bottom]{footmisc}
\bibliographystyle{acm}


\pagestyle{fancy}
\hyphenation{Arc-Sim pack-ages}

%% space between text and footnote
\makeatletter
\renewcommand\footnoterule{%
  \vspace{1em}%   <-- one line space between text and footnoterule
  \kern-3\p@\hrule\@width.4\columnwidth%
  \kern2.6\p@}
\makeatother

%% By default the TOC has only dotted lines between title and pagenumber for section and lower. 
%% This also defines dotted lines for chapters.
\makeatletter
\renewcommand\l@chapter[2]{%
  \ifnum \c@tocdepth >\m@ne
    \addpenalty{-\@highpenalty}%
    \vskip 1.0em \@plus\p@
    \setlength\@tempdima{1.5em}%
    \begingroup
      \parindent \z@ \rightskip \@pnumwidth
      \parfillskip -\@pnumwidth
      \leavevmode
      \advance\leftskip\@tempdima
      \hskip -\leftskip
      {\bfseries #1}\nobreak\leaders\hbox{$\m@th \mkern \@dotsep mu.\mkern
	 \@dotsep mu$}\hfill \nobreak
             \hbox to\@pnumwidth{\hss \bfseries #2}\par
      \penalty\@highpenalty
    \endgroup
  \fi} 
\makeatother

%%%%%%%%%%%%%%%%%%%%%%%%%%%%
%  Fancyheader definition  %
%%%%%%%%%%%%%%%%%%%%%%%%%%%%
\setlength{\headheight}{30pt}
\fancyhf{}
\renewcommand{\sectionmark}[1]{\uppercase{\markboth{#1}{#1}}}
\lhead{\leftmark}
\rhead{\thepage}
\renewcommand{\headrulewidth}{0.5pt}

%%%%%%%%%%%%%%%%%%%%%%%%%%%%
%  Fancyfooter definition  %
%%%%%%%%%%%%%%%%%%%%%%%%%%%%
\setlength{\footskip}{30pt}
\fancyfoot{}
\renewcommand{\footrulewidth}{0.5pt}

%%%%%%%%%%%%%%%%%%%%%%%%%%%%%%%%%%
%  If pagestyle plain is called  %
%%%%%%%%%%%%%%%%%%%%%%%%%%%%%%%%%%
\fancypagestyle{plain}{%
\fancyhead{}
\fancyhead[OR]{\thepage}
\fancyfoot{}
}

%%%%%%%%%%%%%%%%%%%%%%%%%%
%  Syntaxbox definition  %
%%%%%%%%%%%%%%%%%%%%%%%%%%
\fboxsep1.36mm%
\definecolor{g1}{gray}{.92}%
\newsavebox{\syntaxbox}%
\newenvironment{sybox}%
{\noindent \begin{lrbox}{\syntaxbox}%
\begin{minipage}{12.3cm}}%
{\end{minipage}%
\end{lrbox}%
{\fcolorbox{g1}{g1}%
{\parbox{12.3cm}{\usebox{\syntaxbox}\hfill\hbox{}}}}\newline\newline}

%%%%%%%%%%%%%%%%%%%%%%%%%%%%%%%%%
%  Syntaxbox header definition  %
%%%%%%%%%%%%%%%%%%%%%%%%%%%%%%%%%
\definecolor{g2}{gray}{.52}%
\newsavebox{\syntaxboxh}%
\newenvironment{syboxh}%
{\noindent \begin{lrbox}{\syntaxboxh}%
\begin{minipage}{12.3cm}}%
{\end{minipage}%
\end{lrbox}%
{\fcolorbox{g2}{g2}%
{\parbox{12.3cm}{\usebox{\syntaxboxh}\hfill\hbox{}}}}\newline}

%%%%%%%%%%%%%%%%%%%%%%%%%%%%%%%%%%%%%%%%%%%%%%%%%%%%%%%%%%%%%%%%%%%%%%%%%%%%%%%%
%%%%%%%%%%%%%%%%%%%%%%%%%%%%%%%%%%%%%%%%%%%%%%%%%%%%%%%%%%%%%%%%%%%%%%%%%%%%%%%%
%%%%%%%%%%%%%%%%%%%%%%%%%%%%%%%%%%%%%%%%%%%%%%%%%%%%%%%%%%%%%%%%%%%%%%%%%%%%%%%%
\begin{document}


%%%%%%%%%%%%%%%
%  TITLEPAGE  %
%%%%%%%%%%%%%%%
\begin{titlepage}
\begin{center}
\begin{figure}[th]
\centering
{\includegraphics[width=0.5\textwidth]{figures/arcsim_logo_no_text-grey}}\\ 
\end{figure}
\vspace{1.5cm}
% TITLE
{\Huge{\textbf{\textsc{ArcSim} Documentation}}}\\
{\small{\textbf{Flexible Ultra-High Speed Instruction Set Simulator}}}\\
\vspace{2cm}
{\small{\textsc{ArcSim} Documentation}}\\
{\small{The University of Edinburgh}}\\
{\small{\the\year}}
\vspace{5cm}
\end{center}
\begin{tabular}{ll}
Version: & \the\year.1\\
Date: & \today\\
Status: & draft\\
Confidentiality: & confidential\\
Reference & ArcSim-sys-doc\\
\end{tabular}

\end{titlepage}
\cleardoublepage

%%%%%%%%%%%%%%%%%%%%%%%%%%%%%%%%%%%%
%
% Titlepage
%
%%%%%%%%%%%%%%%%%%%%%%%%%%%%%%%%%%%% 
\newpage
\begingroup
\renewcommand*{\addvspace}[1]{}
\renewcommand{\contentsname}{Table of Contents}
\tableofcontents
\newpage
\endgroup
\pagenumbering{arabic} \setcounter{page}{1}
%%%%%%%%%%%%%%%%%%%%%%%%%%%%%%%%%%%%%%%%%%%%%%%%%%%%%%%%%%%%%%%%%%%
% Leave some space between paragraphs
\setlength{\parskip}{.1in}
% Do not indent paragraphs
\setlength{\parindent}{0in}
%%%%%%%%%%%%%%%%%%%%%%%%%%%%%%%%%%%%%%%%%%%%%%%%%%%%%%%%%%%%%%%%%%%

%%%%%%%%%%%%%%%%%%%%%%%%%%%%%%%%%%%%
%
% Inroduction 
%
%%%%%%%%%%%%%%%%%%%%%%%%%%%%%%%%%%%%
\chapter[Introduction]{Introduction}

The \textsc{ArcSim} simulator is a highly configurable ultra-high speed
Instruction Set Simulator (ISS).  Architectural features such as register file
size, instruction set extensions, the set of branch conditions, the auxiliary
register set, as well as memory mapped \textsc{Io} and closely coupled memory
(\textsc{Ccm}) extensions can be specified via a set of well defined \textsc{Api}s
and configuration settings. Furthermore, microarchitectural features such as pipeline
depth, per instruction execution latencies, cache size and associativity, cache block
replacement policies, memory subsystem layout, branch prediction strategies, as
well as bus and memory access latencies are fully configurable.

Overall the simulator provides the following simulation modes:
\begin{itemize}
  \item \emph{Co-simulation} mode working in lock-step with standard hardware
    simulation tools used for hardware and performance verification.

  \item Highly-optimised \emph{interpretive} simulation mode.

  \item \emph{High-speed} \textsc{Dbt} simulation mode capable of simulating an
  embedded system at speeds approaching or even exceeding that of a silicon
  \textsc{Asip} whilst faithfully modelling the processor's architectural state.

  \item \emph{High-speed} target microarchitecture adaptable
  \emph{cycle-accurate} simulation mode modelling the processor pipeline,
  caches, and memories.

  \item A \emph{profiling} simulation mode that is orthogonal to the above
  modes delivering additional statistics such as dynamic instruction
  frequencies, detailed per register access statistics, per instruction
  latency distributions, detailed cache statistics, executed delay slot
  instructions, as well as various branch predictor statistics.

\end{itemize}

%%%%%%%%%%%%%%%%%%%%%%%%%%%%%%%%%%%%
%
% \textsc{ArcSim} Software Design and Architecture
%
%%%%%%%%%%%%%%%%%%%%%%%%%%%%%%%%%%%%
\chapter[Software Architecture]{Software Architecture}

\section{Overview}

\textsc{ArcSim} is a target adaptable \textsc{Iss} providing full
support of the \textsc{Arc}ompact\textsuperscript{\texttrademark} \textsc{Isa}.
It is a full-system simulator, implementing the processor, its memory sub-system
(including \textsc{Mmu}, memory mapped \textsc{Io} devices, closely coupled
memories, etc.), and sufficient interrupt-driven peripherals to simulate the
boot-up and interactive operation of a complete Linux-based system.

\textsc{ArcSim} implements state-of-the-art just-in-time (\textsc{Jit}) dynamic
binary translation (\textsc{Dbt}) techniques, combining interpretive and compiled
simulation techniques in order to maintain high speed, observability and flexibility.

\subsection{Dynamic Binary Translation}

Efficient \textsc{Dbt} heavily relies on Just-in-Time (\textsc{Jit})
compilation for the translation of target machine instructions to host
machine instructions. Although \textsc{Jit} compiled code generally runs
much faster than interpreted code, \textsc{Jit} compilation incurs an
additional overhead. For this reason, only the most frequently executed code
regions are translated to native code whereas less frequently executed code is
still interpreted. Using a single-threaded execution model, the interpreter pauses
until the \textsc{Jit} compiler has translated its assigned code block and the
generated native code is executed directly. But program execution does not need to be
paused to permit compilation, as a \textsc{Jit} compiler can operate
in a separate thread while the program executes concurrently. This {\em
  decoupled} or {\em asynchronous} execution of the \textsc{Jit}
compiler increases complexity of the \textsc{Dbt}, but is very
effective in hiding the compilation latency -- especially if the
\textsc{Jit} compiler can run on a separate processor.

\textsc{ArcSim} implements state-of-the-art \textsc{Jit} dynamic binary translation
capable of effectively reducing dynamic compilation overhead, yielding execution
speedups by doing \emph{parallel} \textsc{Jit} compilation, exploiting the broad
proliferation of multi-core processors. The key idea is to detect {\em
independent}, large translation units in execution traces and to {\em farm out}
work to {\em multiple, concurrent} \textsc{Jit} compilation workers. To ensure
that the latest and most frequently executed code traces are compiled first, we
apply a priority queue based dynamic work scheduling strategy where the most
recent, hottest traces are given highest priority.


%% FIGURE: JIT compilation flow ------------------------------------------------
\begin{figure}
\centering
  \includegraphics[width=0.99\textwidth]{figures/jit-translation-flow-chart} 
  \caption
    {\textsc{Jit} Dynamic Binary Translation Flow.}
  \label{fig:jit-flow} 
\end{figure}
% ------------------------------------------------------------------------------




\section{Simulation Modes}

\subsection{Architectural Simulation}

\subsection{Microarchitectural Simulation}

\subsection{Co-Simulation}

\subsection{Profiling Simulation}

\subsection{Dynamic Binary Translation}


%%%%%%%%%%%%%%%%%%%%%%%%%%%%%%%%%%%%
%
% Using ArcSim
%
%%%%%%%%%%%%%%%%%%%%%%%%%%%%%%%%%%%%
\chapter[Using \textsc{ArcSim}]{Using \textsc{ArcSim}}

\section{Overview}

\section{Configuration}

\section{Running \textsc{ArcSim}}

\begin{syboxh}
\textbf{Running a Simulation}
\end{syboxh}
\begin{sybox}
\scriptsize{\texttt{%
\$ arcsim -e binary.x
}}
\end{sybox}



%%%%%%%%%%%%%%%%%%%%%%%%%%%%%%%%%%%%
%
% Integrating ArcSim
%
%%%%%%%%%%%%%%%%%%%%%%%%%%%%%%%%%%%%
\chapter[Integrating \textsc{ArcSim}]{Integrating \textsc{ArcSim}}

\section{Overview}

\section{Application Programming Interface}


%%%%%%%%%%%%%%%%%%%%%%%%%%%%%%%%%%%%
%
% LIST OF FIGURES
%
%%%%%%%%%%%%%%%%%%%%%%%%%%%%%%%%%%%% 
\begingroup
\clearpage
\setlength{\parskip}{0in}
\newpage
\listoffigures
\newpage
\endgroup

\end{document}
